\documentclass{article}

\usepackage{graphicx}
\usepackage{float}
\usepackage{fancyvrb}
\usepackage[T1]{fontenc}
\usepackage{lmodern}
\usepackage{setspace}
\usepackage[nottoc]{tocbibind}
\usepackage[font=large]{caption}
\usepackage{framed}
\usepackage{tabularx}
\usepackage{amsmath}
\usepackage{hyperref}
\usepackage{fontspec}
\usepackage[backend=biber,sorting=none]{biblatex}
%%\usepackage[
%%	backend=biber,
%%	style=ieee,
%%	sorting=none
%%]{biblatex}
%\addbibresource{project_refs.bib}

%% Hide section, subsection, and subsubsection numbering
%\renewcommand{\thesection}{}
%\renewcommand{\thesubsection}{}
%\renewcommand{\thesubsubsection}{}

% Alternative form of doing section stuff
\renewcommand{\thesection}{}
\renewcommand{\thesubsection}{\arabic{section}.\arabic{subsection}}
\makeatletter
\def\@seccntformat#1{\csname #1ignore\expandafter\endcsname\csname the#1\endcsname\quad}
\let\sectionignore\@gobbletwo
\let\latex@numberline\numberline
\def\numberline#1{\if\relax#1\relax\else\latex@numberline{#1}\fi}
\makeatother

\makeatletter
\renewcommand\tableofcontents{%
    \@starttoc{toc}%
}
\makeatother

\newcommand{\respacing}{\doublespacing \singlespacing}
\newcommand{\code}[2][1]{\noindent \texttt{\tab[#1] #2} \\}
\newcommand{\skipline}[0]{\texttt{}\\}
\newcommand{\tnl}[0]{\tabularnewline}

\begin{document}
%--------
	\font\titlefont={Times New Roman} at 20pt
	\title{{\titlefont Fling HDL Language Implementation Notes }}

	\font\bottomtextfont={Times New Roman} at 12pt
	\date{{\bottomtextfont} \today}
	\author{{\bottomtextfont Andrew Clark}}

	\setmainfont{Times New Roman}
	\setmonofont{Courier New}

	\maketitle
	\pagenumbering{gobble}

	\newpage
	\pagenumbering{arabic}
	%\tableofcontents
	%\newpage

	\doublespacing

%\section{Abstract}
	%\setcounter{section}{-1}

\section{Table of Contents}
	\tableofcontents
	\newpage

\section{Components}
	\subsection{Phase:  Parsing to create initial AST}
		\begin{itemize}
		%--------
		\item Parsing itself is done with an ANTLR4-generated parser.
		\item A parse tree visitor, \texttt{PtVisitor}, is used to
			construct the initial AST.  This is largely a finished part of
			the compiler, not counting any bugs that may remain, and it is
			ready to be expanded with future language constructs.
		\item This stage does \textit{not} create the symbol table.
			\begin{itemize}
			%--------
			\item Note:  the AST and symbol table are considered immutable,
				so changes to the structure of the AST and symbol table
				will simply be provided by generating new ones, generating
				the AST first, followed by a new symbol table created from
				that AST.
			\item Note:  The symbol table and AST are the two primary data
				structures within the compiler, besides anything needed for
				running the partial language interpreter.  (One day there
				may be a full language interpreter).
			%--------
			\end{itemize}
		%--------
		\end{itemize}
	\subsection{Stage:  Create Symbol Table }
		\begin{itemize}
		%--------
		\item This stage is done after every generation of a new AST, and
			it is performed with an AST visitor.  The reasoning for doing
			this separately is that it ensures consistent generation of the
			symbol table for a particular AST, i.e. it localizes bugs in
			code that generates the symbol table.
		\item This stage takes multiple passes because ordering of symbol
			definitions does not matter per the language's semantics.
		\item This stage does some semantic analysis.  It prevents multiple
			instances of the same symbol identifier within one scope.
		\item This stage does \textit{not} expand template instances into
			their own symbols.
		\item This stage does \textit{not} elaborate the values of named
			constants.
			\begin{itemize}
			%--------
			\item Note:  Once named constants are evaluated, it is no longer
				necessary to compute their values.  It makes sense to copy
				their values from one symbol table to the next.  Some
				mechanism needs to exist to facilitate this goal.
			\item Note:  Named constants will need to be translated into
				Verilog's \texttt{localparam}s, which are always Verilog
				vectors.  Non-\texttt{packed} \texttt{class}es and arrays
				in my language will necessarily be translated freely into
			%--------
			\end{itemize}
		\item This stage will handle the semantics of \texttt{class}es and
			\texttt{mixin}s, including inheritance.
			\begin{itemize}
			%--------
			\item But what about multiple inheritance of mixins anyway?
				How should that be set up?  Problems only exist when names
				clash.
			\item Here is my idea, based on how Python does things:  have
				name clashes be resolved by simply using whatever showed up
				first within the \texttt{extends} list, prioritizing an
				inherited \texttt{class}'s named object over an inherited
				\texttt{mixin}'s.
			%--------
			\end{itemize}
		%--------
		\end{itemize}
	\subsection{Phase:  First AST lowering phase}
		This phase will perform each of the following, as much as can be
		accomplished, on every other pass.  The other passes are used for
		building a new symbol table.
		\begin{itemize}
		%--------
		\item Expand instantiated templates and \texttt{parpk} into new AST
			nodes.  This is to be done for \texttt{module}s, template types,
			and subprograms.
		\item Elaborate \texttt{gen} constructs into new AST nodes.
		\item Flatten \texttt{proc} calls into new AST nodes.  Since
			this is part of the metaprogramming system, I believe it
			can go here.
		\item Use the \texttt{Interpreter} class to compute the values of
			constants as needed for the previous items.
		%--------
		\end{itemize}
		Note that any of these items besides buiding a new symbol table for
		the new AST will involve giving up the current node's translation
		if it depends on another stage within this phase.  Also, each of
		these items will build a new symbol table upon their (potentially
		partial) completion.


		This phase should inject information about where new AST nodes came
		from for error messages:
		\begin{itemize}
		%--------
		\item Example:  \texttt{Error:  in expansion of proc
			build\_adder()...}
		%--------
		\end{itemize}

		%Every pass of this section of this phase will rely upon the one
		%always following it, another Create Symbol Table pass, to determine
		%if the expansion created too many scopes.  

		%I believe that circular dependencies can be prevented by not
		%allowing a template to depend on another instance of itself.  Does
		%this break anywhere?  I don't think I've ever written C++ code
		%where a template dependended upon itself.  This seems both
		%reasonable and to be the solution.

		\subsubsection{Handling Circular Dependencies}
		Here are some notes on handling circular dependencies within this
		phase.
		\begin{itemize}
		%--------
		\item A template \texttt{class} (or \texttt{mixin}) that takes
			an instance of itself as a template parameter can simply be
			banned.
			%\begin{itemize}
			%%--------
			%\item Note:  this won't mean that a \texttt{type} or
			%	\texttt{module} template parameter can't be used.
			%%--------
			%\end{itemize}
		\item Allow \texttt{class}es, \texttt{mixin}s, and
			\texttt{module}s to contain instances of themselves,
			recursively, but only up to a maximum recursion depth.  A
			\texttt{gen if} construct can be used to determine
			if there should be recursion.  I only want this because even
			\textit{Verilog} allows recursive \texttt{module}s via this
			method.
			\begin{itemize}
			%--------
			\item How should recursion be handled?  How about implementing
				this check for recursion as based upon the specific type of
				module (same parameters and everything), limiting the
				amount of recursion in those.  That can simply be
				implemented with an \texttt{std::map<Symbol*, size\_t>}.
				If any of the \texttt{Symbol} pointers have too many
				instances due to recursion, we throw an error.
			%--------
			\end{itemize}
		\item Of course, don't allow classes or mixins to derive from their
			own type.
			\begin{itemize}
			%--------
			\item This can probably be handled by just using an
				\texttt{std::set<Symbol*>}.
			%--------
			\end{itemize}
		%--------
		\end{itemize}

		%\subsubsection{Algorithms For This Stage}
		%	Template instance expansion algorithm:
		%	\begin{itemize}
		%	%--------
		%	\item For every template instance:
		%		\begin{itemize}
		%		%--------
		%		\item If this template instance has no corresponding AST
		%			node in the new AST (kept track of with some kind of
		%			\texttt{std::map})

		%			\begin{itemize}
		%			%--------
		%			\item Create a new AST node representing this
		%				template instance.
		%			%--------
		%			\end{itemize}
		%		%--------
		%		\end{itemize}
		%	%--------
		%	\end{itemize}


	%\printbibliography[heading=bibnumbered,title={Bibliography}]

\end{document}
