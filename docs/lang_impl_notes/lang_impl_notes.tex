\documentclass{article}

\usepackage{graphicx}
\usepackage{float}
\usepackage{fancyvrb}
\usepackage[T1]{fontenc}
\usepackage{lmodern}
\usepackage{setspace}
\usepackage[nottoc]{tocbibind}
\usepackage[font=large]{caption}
\usepackage{framed}
\usepackage{tabularx}
\usepackage{amsmath}
\usepackage{hyperref}
\usepackage{fontspec}
\usepackage[backend=biber,sorting=none]{biblatex}
%%\usepackage[
%%	backend=biber,
%%	style=ieee,
%%	sorting=none
%%]{biblatex}
%\addbibresource{project_refs.bib}

%% Hide section, subsection, and subsubsection numbering
%\renewcommand{\thesection}{}
%\renewcommand{\thesubsection}{}
%\renewcommand{\thesubsubsection}{}

% Alternative form of doing section stuff
\renewcommand{\thesection}{}
\renewcommand{\thesubsection}{\arabic{section}.\arabic{subsection}}
\makeatletter
\def\@seccntformat#1{\csname #1ignore\expandafter\endcsname\csname the#1\endcsname\quad}
\let\sectionignore\@gobbletwo
\let\latex@numberline\numberline
\def\numberline#1{\if\relax#1\relax\else\latex@numberline{#1}\fi}
\makeatother

\makeatletter
\renewcommand\tableofcontents{%
    \@starttoc{toc}%
}
\makeatother

\newcommand{\respacing}{\doublespacing \singlespacing}
\newcommand{\code}[2][1]{\noindent \texttt{\tab[#1] #2} \\}
\newcommand{\skipline}[0]{\texttt{}\\}
\newcommand{\tnl}[0]{\tabularnewline}

\begin{document}
%--------
	\font\titlefont={Times New Roman} at 20pt
	\title{{\titlefont Fling HDL Language Implementation Notes }}

	\font\bottomtextfont={Times New Roman} at 12pt
	\date{{\bottomtextfont} \today}
	\author{{\bottomtextfont Andrew Clark}}

	\maketitle
	\pagenumbering{gobble}

	\newpage
	\pagenumbering{arabic}
	\tableofcontents

	\newpage
	\setmainfont{Times New Roman}
	\setmonofont{Courier New}

%\doublespacing
%\section{Abstract}
	%\setcounter{section}{-1}

\section{Table of Contents}
	\tableofcontents
	\newpage

\section{Passes}
	\begin{enumerate}
	%--------
	\item Parsing to create initial AST
		\begin{itemize}
		%--------
		\item Parsing itself is done with an ANTLR4-generated parser.
		\item A parse tree visitor, \texttt{PtVisitor}, is used to
		construct the initial AST.  This is largely done, not counting any
		bugs that may remain, and it is ready to be expanded with future
		language constructs.
		\item This stage does \textit{not} create the symbol table.
		\item Note:  the AST and symbol table are considered immutable, so
		changes to the structure of the AST and symbol table will simply be
		provided by generating new ones, generating the AST first, followed
		by a new symbol table created from that AST.
		\item Note:  The symbol table and AST are the two primary data
		structures within the compiler, besides anything needed for running
		the partial language interpreter.  (One day there may be a full
		language interpreter).
		%--------
		\end{itemize}
	\item Create Symbol Table (Same algorithm used for all cases when a new
	symbol table is built)
		\begin{itemize}
		%--------
		\item This is performed with an AST visitor.
		\item This stage takes multiple passes because ordering of symbol
		definitions does not matter per the language's semantics.
		\item This stage does some semantic analysis.  It prevents
		multiple instances of the same symbol identifier within one scope.
		\item This stage should \textit{not} expand template instances into
		their own symbols.  Let's leave that for the next phase of the
		compiler.
		\item Note:  Once named constants are evaluated, it is no longer
		necessary to compute their values.  It makes sense to transfer
		their values from one symbol table to the next.  Some mechanism
		needs to exist to facilitate this goal.
		%--------
		\end{itemize}
	\item First phase of AST lowering.
		\begin{itemize}
		%--------
		\item This stage needs to perform the following, not necessarily in
		this order.
			\begin{itemize}
			%--------
			\item Expand template instances into new AST nodes.  This is to
			be done for \texttt{module}s, types, and subprograms.
			\item Elaborate \texttt{gen} constructs into new AST nodes.
			\item Elaborate \texttt{proc}s into new AST nodes when called
			inside of \texttt{module}s.  Since this is part of the
			metaprogramming system, I believe it can go here.
			\item Compute the values of constants as needed for the
			previous items.  A template that takes a \texttt{size\_t}
			parameter will need to be elaborated with whatever value was
			provided.  A \texttt{gen} \texttt{for} construct will need to
			know the range of iteration to be performed.
				\begin{itemize}
				%--------
				\item This item needs an interpreter for the components of
				the language that can be evaluated at compile-time.
				\item Also, the interpreter itself may find a template
				instance or a \texttt{gen} construct that itself needs
				expanded.
				%--------
				\end{itemize}
			%--------
			\end{itemize}
		\item This phase should inject information about where new AST
		nodes came from for error messages.  (\texttt{In
		expansion of proc build\_adder(), filename.fling\_hdl:30:60})
		\item Every pass of this section of this phase will rely upon the
		one always following it, another Create Symbol Table pass, to
		determine if the expansion created too many scopes.  
		\item I believe that circular dependencies can be prevented by not
		allowing a template to depend on another instance of itself.  Does
		this break anywhere?  I don't think I've ever written C++ code
		where a template dependended upon itself.  This seems both
		reasonable and to be the solution.
			\begin{itemize}
			%--------
			\item A template \texttt{class} (or \texttt{mixin}) that takes
			an instance of itself as an argument can simply be banned.
			\item Allow \texttt{class}es, \texttt{mixin}s, and
			\texttt{module}s to contain instances of themselves,
			recursively, but only up to a maximum recursion depth.  A
			\texttt{gen} \texttt{if} construct can be used to determine if
			there should be recursion.  Verilog already allows recursive
			\texttt{module}s via this method.
			%--------
			\end{itemize}
		%--------
		\end{itemize}
	%--------
	\end{enumerate}


	%\printbibliography[heading=bibnumbered,title={Bibliography}]

\end{document}
